\section{Introduction}
\label{sec:intro}

Vaccination and social distancing are the primary strategies for controlling the spread of epidemic outbreaks
\cite{medlock:science09,Ogura2017,halloran:pnas08,lofgren:pnas14,zhang2015controlling,YaoSDM2014,AAAI1816714,PreciadoVM13_2,PreciadoVM13,PreciadoVM14,Aspnes:2005}.
The production of vaccines is expensive and time intensive, and so there is always a shortage of
vaccine supply \cite{cdc:temporal}. As a result, there is a lot of interest in evaluating different kinds of
interventions \cite{halloran:pnas08,lofgren:pnas14}, and finding optimal interventions \cite{medlock:science09}.
The spread of epidemics is very complex, and SIR type diffusion processes (and variations) are commonly used:
Informally, each infected individual $u$ (in state I) spreads the infection to each susceptible neighbor $v$ 
(in state S) with probability $p(u, v)$; this is defined formally in Section \ref{sec:prelim}.
These can be modeled as system of differential equations \cite{medlock:science09,AAAI1816714,venkataramanan:ichi17}, or as stochastic
agent based models on social contact networks, e.g., \cite{marathe:cacm13}.
Differential equation models are small enough that they can be solved optimally by simple brute-force
local search methods \cite{medlock:science09}.

However, network and agent based models cannot be easily optimized this way. 
In this paper, we study problem \prob{} of designing temporal vaccination strategies to minimize
the expected outbreak size in an SIR epidemic process on a network $G=(V, E)$.
There is a lot of relevant prior work on this topic, and can be split along the following lines
(we only consider work restricted to network models):
(1) Optimization in the SI/SIS type models, in static ot dynamic networks, e.g.,
\cite{PreciadoVM13_2,PreciadoVM13,PreciadoVM14,SahaSDM15,Ogura2017}. Much of this work is based on
controlling spectral properties, but does not give any guaranteed bounds on the expected outbreak size;
(2) Static interventions in SIR models, e.g., \cite{zhang2015controlling,YaoSDM2014}, which also do not
directly bound the expected outbreak size,
(3) SI model with transmission probability 1 \cite{Aspnes:2005}, which does not translate to results for
the SIR model with lower transmission probability;
(4) Heuristics for picking nodes based on degree or centrality, e.g., \cite{havlin},
which work for all models but give no guarantees.
In summary, none of the prior results give rigorous results for \prob{}, which remains an open problem.

\noindent
\textbf{Our results.}
\emph{We present the first approximation algorithm for \prob{} with rigorous guarantees.}
Our specific contributions are described below.
\begin{itemize}
\item
We design algorithm \algo{} for selecting the sets of nodes to vaccinate at different times,
within a given set of temporal budget constraints. We show a rigorous worst case approximation guarantee on
the performance of \algo{}, which is logarithmic in the number of paths in sampled subgraphs of $G$;
this is typically significantly smaller than the number of paths in $G$, so that
in practice, \algo{} has a much smaller approximation ratio.
Our main technical ideas are rounding a linear programming (LP) relaxation, along with the sample average technique.
\item
We augment \algo{} with a sparsification step, which significantly reduces the size of the LP, and
is able to scale to networks with millions of nodes and edges.
\item 
We evaluate our algorithms on diverse real and random networks.
We show that \algo{} has an approximation ratio very close to $1$, significantly better than the worst case guarantee we prove rigorously. 
We find that \algo{} outperforms two of the most commonly used baselines for intervention design.
We also examine the structure of solutions, and find significant differences in the characteristics of nodes
picked in different stages.
\end{itemize}

\noindent
\textbf{Public health impact.}
At the start of every flu outbreak, and during major pandemics, there is a lot of interest from the CDC and other public health agencies on finding optimal solutions in different models, e.g., \cite{medlock:science09,halloran:pnas08,lofgren:pnas14,venkataramanan:ichi17}. These can be used to guide policies when there are shortages, including the logistics of where vaccines should be deployed \cite{venkataramanan:ichi17}. Our methods help designing near-optimal policies using agent based models, which have been found to be more useful in planning for large outbreaks, e.g., \cite{halloran:pnas08,lofgren:pnas14,eubank:nature04,gk06}.

%%%In most diseases, e.g. flu, the final policies have to be \emph{implementable}. This is ill-defined, but one characteristic is that the interventions should not target specific individuals, but could be defined by broader categories, e.g., age (as in the CDC guidelines). In such a case, the solution from a method like ours will not translate into an implementable policy. However, our algorithms can be used for a group level intervention design problem, as in \cite{zhang2015controlling}, which would give more implementable solutions.

