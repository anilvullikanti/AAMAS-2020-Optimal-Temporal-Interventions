\section{Related work}
\label{sec:related}

\subsection{Public health policy planning and use of diffusion models}
Public health policy analysis relies heavily on mathematical models of SIR type processes, e.g., \cite{lofgren:pnas14,anderson+m:book}. As discussed earlier, there are two broad classes of models. The first involves using a system of coupled differential equations to represent the dynamics, e.g., \cite{medlock:science09,AAAI1816714,venkataramanan:ichi17}. These do not have any closed form solutions, in general, and when the system is not very large, it can be solved by brute force local search methods \cite{medlock:science09}. For some types of models, greedy strategies have been used \cite{AAAI1816714,venkataramanan:ichi17}. 
The second is network based, and uses a stochastic diffusion model for the spread of the disease \cite{marathe:cacm13,halloran:pnas08,lofgren:pnas14,eubank:nature04,gk06}. Such models have been found to be more powerful and useful for epidemic spread on large heterogeneous populations, where the complete mixing assumptions of differential equation models are not valid.

During any large outbreak, public health agencies solve a variety of models, and make plans and guidelines based on the results,
e.g., during the 2009 swine flu \cite{medlock:science09}, and
the 2014 Ebola outbreak \cite{lofgren:pnas14}. Such studies typically explore the space of different possible interventions, within given resource constraints. Therefore, there is a lot of interest in the design of optimal or near-optimal interventions.

\subsection{Resource optimization to control epidemic spread}
There is a lot of work on this topic, and we summarize the main research directions relevant to \prob{}.

\noindent
\textbf{Firefighter problems}, e.g., \cite{anshelevich09,Finbow2009TheFP,Chalermsook:2010:RMF:1873601.1873709}. 
This is most closely related to \prob{}. The basic version of the problem is to
determine a temporal intervention $\X=\{X_t: t=1,\ldots,n\}$, such that $|\X_t|\leq B$,
and the number of nodes not infected (i.e., saved) is maximized (this is referred to as the Max-Save version).
The disease model is a SI process with $p=1$ (so this is a deterministic model).
The Max-Save version cannot be approximated within an $\Omega(n^{\epsilon})$ factor for any $\epsilon < 1$, in general graphs.
A related problem is the Min-Budget version, for which an $O(\sqrt{n})$ approximation is 
possible \cite{Chalermsook:2010:RMF:1873601.1873709}.
This work corresponds to \emph{non-spreading} interventions. Better approximations are possible for the setting in which
the vaccination is also a \emph{spreading} process.
We refer to Finbow et al. \cite{Finbow2009TheFP} for an extensive survey on the firefighter problems.
One of the few works on the firefighter problem with a stochastic disease model (i.e., $p<1$) is by
Tennenholtz et al. \cite{DBLP:journals/corr/abs-1711-08237}, who formalize the problem as a Markov Decision Process (MDP),
and compute an optimal solution for trees.  One of the main differences between this and our paper is that
the MDP formulation of \cite{DBLP:journals/corr/abs-1711-08237} makes the problem \emph{adaptive},
i.e., it is possible to use information about which nodes are infected before time $t$, in designing 
the intervention $\X_t$. In contrast, \prob{} only considers a non-adaptive setting, and the intervention
has to be determined ahead of time.

\noindent
\textbf{Optimization of spectral properties.}
A key result in epidemic modeling is a characterization of an outbreak in terms of spectral properties,
namely the first eigenvalue (also referred to as the spectral radius, and denoted by $\lambda_1$)

\cite{PreciadoVM13_2,PreciadoVM13,PreciadoVM14,SahaSDM15,Ogura2017}. Much of this work is based on
controlling spectral properties, but does not give any guaranteed bounds on the expected outbreak size;
(2) Firefighter problem, which can be viewed as \prob{} on SI model with with $p=1$, e.g.,
\cite{anshelevich09,Finbow2009TheFP}. Rigorous bounds are known for the number infected and saved.
However, this has not been studied much for the case where $p<1$, except \cite{DBLP:journals/corr/abs-1711-08237},
which only gives rigorous algorithms for trees.
A special case of this problem is with work of \cite{Aspnes:2005}, which considers \prob{} but with
the intervention specified at time 0;
(3) Static interventions in SIR models, e.g., \cite{zhang2015controlling,YaoSDM2014}, which also do not
directly bound the expected outbreak size,
(4) Heuristics for picking nodes based on degree or centrality, e.g., \cite{PhysRevLett.91.247901,Miller2007EffectiveVS},
which work for all models but give no guarantees.
In summary, none of the prior results give rigorous results for \prob{}, to minimize the expected
number of infections, with budget constraints over time, in SIR models of epidemics, with transmission probability less than $1$.


There has been a lot of work on optimizing vaccine allocation. A lot of it has been based on the idea of picking nodes to vaccinate, based on their eigenscore, i.e., their component values in the first eigenvector \cite{PreciadoVM13_2,PreciadoVM13,PreciadoVM14,SahaSDM15,Ogura2017,zhang2015controlling,YaoSDM2014}. It is not clear how to incorporate temporal constraints and delay and information within such formulations. In particular, these results do not incorporate the effects of where the sources are, and instead try to shift the phase transition for a large outbreak, which is in terms of the spectral radius. As we have discussed, such static interventions don't do very well.


