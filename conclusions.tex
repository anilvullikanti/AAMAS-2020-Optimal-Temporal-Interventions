\section{Conclusions}
\label{sec:conc}

We develop the first rigorous algorithm for temporal intervention design to control the spread of epidemics on networks. Our algorithm performs significantly better than standard baselines, and has near-optimal approximation guarantee in practice. Our main techniques are linear programming based rounding and the sample average approximation technique.
We find that the temporal dimension leads to significant changes in the solution quality and structure. Finally, we study the impact of delay and intervention time. We find that in some networks, there is a point up to which waiting helps. Our methods can help in public health policy planning and response to large outbreaks. 
